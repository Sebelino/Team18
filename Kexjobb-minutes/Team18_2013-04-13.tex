% @author Sebastian Olsson
%
% Anv: 294 142 104
% Psw: 2nzp19
%
% quad-bild i team18/kivy/untitled.png
% activate-grej i program files/code laboratories/cl-eye platform sdk

\documentclass[a4paper,12pt]{article}
\usepackage[utf8]{inputenc}
\usepackage[swedish]{babel}
\usepackage[parfill]{parskip}

\author{Sebastian Olsson}
\title{DD143X: RADD Startup Phase}
%\date{8 mars 2013}
\begin{document}
\maketitle
\begin{abstract}
    Mötesanteckningar från mötet 130413, 18:00-19:07.
\end{abstract}

\section{Mötet börjar}
Sebastian förklarar mötet öppnat.

Närvarande:
\begin{itemize}
\item Sebastian Olsson
\item Chjun-Chi Chiu
\item Dmitrij Lioubartsev
\end{itemize}
Frånvarande:
\begin{itemize}
\item Kristoffer Hallqvist
\item Andrew Saka
\item Jonatan Åkesson
\end{itemize}

Ordförande: Sebastian Olsson

Sekreterare: Sebastian Olsson

Mötesplats: Skype.

\section{Föregående protokoll}
Referens: Implementation Phase.

\begin{enumerate}
\item Ej verkställd. Henrik jobbar på det ännu.
\item Verkställd.
\item Verkställd.
\item Verkställd.
\item Verkställd.
\item Verkställd.
\end{enumerate}

\section{Diskussionsteman}
\begin{itemize}
\item RADD
\item Fyra kameror
\item Basic-krav i RURD:n
\item GUI:t
\item Definiera egna mappings
\item Scroll/pinch-funktionalitet
\end{itemize}

\section{Beslut}
\begin{enumerate}
\item Vi skippar funktionaliteten att kunna exekvera kommandon medan en gest utförs, åtminstone
tills vidare.
\item Dmitrij forskar på att låta användaren definiera egna gester och koppla dem till
kommandon via GUI:t.
\item De övriga programmerarna jobbar på att koppla samman GUI:t till resten av programmet.
\item Sebastian kontaktar Jonatan om att boka tid med Henrik inför nästa vecka.
\end{enumerate}

\section{Nästa möte}
Vid nästa experimentomgång nästa vecka, i Voljären.
\end{document}
