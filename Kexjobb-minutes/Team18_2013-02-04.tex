% @author Sebastian Olsson
% @version 2013-02-04

\documentclass[a4paper,12pt]{article}
\usepackage[utf8]{inputenc}
\usepackage[swedish]{babel}
\usepackage[parfill]{parskip}

\author{Sebastian Olsson}
\title{DD143X: Startup ADD Phase}
\begin{document}
\maketitle
\begin{abstract}
    Mötesanteckningar från mötet 130204, 17:10-17:20.
\end{abstract}

\section{Mötet börjar}
Sebastian förklarar mötet öppnat.

Närvarande:
\begin{itemize}
\item Sebastian Olsson
\item Kristoffer Hallqvist
\item Dmitrij Lioubartsev
\item Chjun-Chi Chiu
\end{itemize}
Frånvarande:
\begin{itemize}
\item Andrew Saka
\item Jonatan Åkesson
\end{itemize}

Ordförande: Sebastian Olsson

Sekreterare: Sebastian Olsson

Mötesplats: K-huset, utanför föreläsningssalen K1.

\section{Föregående protokoll}
Referens: Experimental Phase.

Punkt 1: Verkställd.

Punkt 2: Verkställd. Enbart Sebastian och Chjun-Chi var fysiskt närvarande.

\section{Diskussionsteman}
\begin{itemize}
\item ADD
\item Att återvända till Voljären
\item Minimalt fungerande applikation (MWE)
\item GUI i Kivy
\item Doodle för möte med Henrik
\item Berghs inlämnade arbete
\item Vem som gör ett MWE
\item Chief programmer-rollen
\end{itemize}

\section{Beslut}
\begin{itemize}
\item Sebastian skapar en Github-repository för projektet.
\item Jonatan kontaktar Henrik om vid vilken tid nästa vecka vi kan använda Voljären.
\item Dmitrij, Kristoffer och Chjun-Chi tilldelas uppgiften att konstruera ett MWE (gärna med
    grafiskt gränssnitt) av en Kivy-applikation som kan användas för skärmen innan veckans slut.
\item Sebastian skapar ett dokument för ADD:n med dess layout.
\end{itemize}

\section{Nästa möte}
Inträffar någon gång nästa vecka (efter mötet med Henrik).
\end{document}
