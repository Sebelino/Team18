% @author Sebastian Olsson
%
% Anv: 294 142 104
% Psw: 2nzp19
%
% quad-bild i team18/kivy/untitled.png
% activate-grej i program files/code laboratories/cl-eye platform sdk

\documentclass[a4paper,12pt]{article}
\usepackage[utf8]{inputenc}
\usepackage[swedish]{babel}
\usepackage[parfill]{parskip}
\usepackage{tikz}

\author{Sebastian Olsson}
\title{DD143X: RADD Finalizing Phase}
\date{22 april 2013}
\begin{document}
\maketitle
\begin{abstract}
    Mötesanteckningar från mötet 130422, 12:00-15:10.
\end{abstract}

\section{Mötet börjar}
Sebastian förklarar mötet öppnat.

Närvarande:
\begin{itemize}
\item Sebastian Olsson
\item Chjun-Chi Chiu
\item Andrew Saka
\end{itemize}
Frånvarande:
\begin{itemize}
\item Kristoffer Hallqvist
\item Dmitrij Lioubartsev
\item Jonatan Åkesson
\end{itemize}

Ordförande: Sebastian Olsson

Sekreterare: Sebastian Olsson

Mötesplats: Voljären.

\section{Föregående protokoll}
Referens: Finalizing the Architecture.

\begin{enumerate}
\item Verkställd.
\item Verkställd.
\end{enumerate}

\section{Diskussionsteman}
\begin{itemize}
\item GUI:t
\item Skript för makron
\item Manualen
\item Kalibrering
\item RADD: arbetsuppdelning och arbetstider
\end{itemize}

\section{Beslut}
\begin{itemize}
\item Sebastian färdigställer kopplingen från GUI:t till databasen (via Controllern).
\item Dmitrij lägger till knappar för att lägga till/ta bort gester och makron från GUI:t.
\item Chjun-chi implementerar åtminstone 10 olika kommandon i CommandHandlern, inklusive:
\begin{itemize}
\item \texttt{open <path>} för att öppna en valfri mapp/program
\item \texttt{keypress <key>} för tangenttryckning
\item \texttt{leftclick} för vänstermusklickning
\item \texttt{rightclick} för högermusklickning
\item \texttt{scroll <px>} för att scrolla uppåt eller nedåt
\item \texttt{sleep <ms>} för att vänta en bestämd tid innan nästa kommando exekveras.
\end{itemize}
\item Jonatan bokar en ny session med Henrik nästa vecka.
\end{itemize}

\section{Nästa möte}
Vid experimentomgången nästa vecka, i Voljären.
\end{document}
